\section{研究总结与展望}

% 第1页:研究总结与未来工作
\begin{frame}{研究总结与未来工作}
\justifying
本研究构建了一个面向水面无人艇(USV)的三层路径规划与控制框架,涵盖:

\begin{itemize}
    \item MD-A* 全局路径规划:结合欧几里得距离与绕行代价的启发式设计;
    \item AN-MPC 局部轨迹控制:基于 Fossen 模型实现动态误差调节与扰动补偿;
\end{itemize}

\vspace{0.5em}
\textbf{后续工作展望:}
\begin{itemize}
    \item 引入多智能体强化学习(MASAC),提升动态障碍规避与协同能力;
    \item 构建强化学习-AN-MPC–MD-A* 联动机制,实现路径规划与控制策略的融合调度;
    \item 拓展系统至真实海洋任务部署,探索稳定性与鲁棒性进一步提升路径。
\end{itemize}
\end{frame}

% 第2页:项目仓库链接(附录B3)
\begin{frame}{项目仓库链接说明}
\justifying
为提升本课题的可复现性与模块化可维护性,作者已将路径规划与控制器模块整理上传至 GitHub,涵盖核心代码、地图数据与实验图像。

\vspace{0.8em}
\textbf{项目地址:}
\begin{center}
\texttt{https://github.com/CGMgit/USV-MD-Astar-ANMPC}
\end{center}

\vspace{0.5em}
\textbf{仓库内容包括:}
\begin{itemize}
    \item \texttt{code/}:MD-A* 与 AN-MPC 实现代码;
    \item \texttt{data/}:仿真地图、节点点位与障碍数据;
    \item \texttt{figures/}:论文与答辩所用实验图;
    \item \texttt{README.md}:项目说明与复现指南。
\end{itemize}

\vspace{0.5em}
项目将在答辩结束后正式开源,欢迎交流与合作。
\end{frame}
