\section{AN-MPC 轨迹控制策略}

% === 第1页:AN-MPC方法的理论设计概述 ===
\begin{frame}{AN-MPC 控制理论基础与设计概述}
    \justifying
    为提升 USV 在复杂扰动环境下的局部轨迹优化能力,本文基于简化 Fossen 模型构建 AN-MPC 控制器,通过滚动优化与动态权重机制实现路径跟踪的稳健性与自适应控制。
    
    \vspace{0.6em}
    USV 离散时间状态空间模型为:
    \[
    \boldsymbol{x}(k+1) = f(\boldsymbol{x}(k), \boldsymbol{u}(k)) + \boldsymbol{w}(k)
    \]
    其中:
    \begin{itemize}
        \item $\boldsymbol{x}(k) = [X, Y, \theta, u, v, \omega]^T$ 为状态向量;
        \item $\boldsymbol{u}(k) = [T_{\text{thrust}}, T_{\text{lateral}}, \tau_{\text{yaw}}]^T$ 为控制输入;
        \item $\boldsymbol{w}(k)$ 表示水流扰动。
    \end{itemize}
    状态预测采用数值积分,考虑扰动叠加。
\end{frame}

% === 第2页:直观思想演示图 ===
\begin{frame}[plain]{直观思想演示图2}
  \begin{center}
    \includegraphics[height=0.7\textwidth]{Image/AN-MPC思想演示图.png}
    \captionof{figure}{2 AN-MPC 轨迹优化思想演示图}
  \end{center}
\end{frame}

% === 第3页:USV 简化动力学建模框架 ===
\begin{frame}{USV 简化动力学建模框架}
\justifying
为实现 AN-MPC 控制器在工程上的可部署性,本文在 Fossen 水动力学模型基础上,进行如下简化处理:
\begin{itemize}
    \item 假设惯性矩阵为对角形式;
    \item 阻尼项采用线性速度相关模型;
    \item 外部扰动建模为定速定向的水流矢量;
\end{itemize}

\vspace{0.5em}
简化后的动力学主方程为:
\[
\boldsymbol{M}\nu + \boldsymbol{C}(\nu)\nu + \boldsymbol{D}(\nu')\nu' = \boldsymbol{\tau}
\]
其中:
\begin{itemize}
    \item $\nu = [u, v, \omega]^T$ 为船体速度向量;
    \item $\boldsymbol{\tau} = [T_{\text{thrust}}, T_{\text{lateral}}, \tau_{\text{yaw}}]^T$ 为控制输入;
    \item $\nu' = \nu - \nu_{\text{current}}^{\text{body}}$ 表示相对于水流的扰动速度。
\end{itemize}
\end{frame}

% === 第4页:水流扰动建模与姿态运动学 ===
\begin{frame}{水流扰动建模与姿态运动学}
\scriptsize
\justifying
水流速度在全局坐标系中为:
\[
\nu_{\text{current}}^{\text{global}} = 
\begin{bmatrix}
V \cos \psi \\ V \sin \psi \\ 0
\end{bmatrix}, \quad
\nu_{\text{current}}^{\text{body}} = R(\theta)^T \nu_{\text{current}}^{\text{global}}
\]

\vspace{0.5em}
三大动力学矩阵定义如下:
\[
\boldsymbol{M} =
\begin{bmatrix}
m & 0 & 0 \\
0 & m & 0 \\
0 & 0 & I_z
\end{bmatrix}, \quad
\boldsymbol{C}(\nu) =
\begin{bmatrix}
0 & 0 & -mv \\
0 & 0 & mu \\
mv & -mu & 0
\end{bmatrix}, \quad
\boldsymbol{D}(\nu') =
\begin{bmatrix}
D_u & 0 & 0 \\
0 & D_v & 0 \\
0 & 0 & D_\omega
\end{bmatrix}
\]

\vspace{0.5em}
USV 姿态变化由运动学方程描述:
\[
\begin{bmatrix}
\dot{X} \\ \dot{Y} \\ \dot{\theta}
\end{bmatrix}
= R(\theta)
\begin{bmatrix} u \\ v \\ \omega \end{bmatrix}
\quad \Rightarrow \quad
\nu_{\text{net}} = \nu_{\text{still}} + \nu_{\text{current}}^{\text{global}}
\]
\end{frame}

% === 第5页:AN-MPC控制目标函数与约束建模 ===
\begin{frame}{AN-MPC 控制目标函数与约束建模}
\justifying
每个控制周期内,AN-MPC 控制器求解如下优化问题:
\[
\min_{\boldsymbol{u}(0),\dots,\boldsymbol{u}(N-1)} \sum_{k=0}^{N-1} \left( (\boldsymbol{x}(k) - \boldsymbol{x}_{\text{ref}}(k))^T Q(k)(\boldsymbol{x}(k) - \boldsymbol{x}_{\text{ref}}(k)) + \boldsymbol{u}(k)^T R \boldsymbol{u}(k) \right)
\]
其中:
\begin{itemize}
    \item $Q(k)$ 和 $R$ 为状态误差与控制能耗的权重矩阵;
    \item $N$ 为预测时域长度,$\boldsymbol{x}_{\text{ref}}(k)$ 为参考轨迹;
    \item $Q(k)$ 动态变化,$R$ 为固定矩阵。
\end{itemize}
\end{frame}

% === 第6页:AN-MPC 多目标代价函数设计 ===
\begin{frame}{AN-MPC 多目标代价函数设计}
\justifying
优化目标函数由多个子目标代价组成:
\[
J = J_{\text{error}} + \lambda_1 J_{\text{energy}} + \lambda_2 J_{\text{speed}} + \lambda_3 J_{\text{curvature}} + \lambda_4 J_{\text{current}} + J_{\text{restricted}}
\]

各子目标定义如下:
\begin{itemize}
    \item 轨迹误差项:$J_{\text{error}} = \sum_k (\boldsymbol{x}_k - \boldsymbol{x}_{\text{ref},k})^T Q (\boldsymbol{x}_k - \boldsymbol{x}_{\text{ref},k})$
    \item 能耗项:$J_{\text{energy}} = \sum_k \boldsymbol{u}_k^T R \boldsymbol{u}_k$
    \item 速度平稳性项:$J_{\text{speed}} = \sum_k \left( (u_k - u_0)^2 + (v_k - v_0)^2 \right)$
\end{itemize}
\end{frame}

% === 第7页:AN-MPC 其他代价与补偿项设计 ===
\begin{frame}{AN-MPC 其他代价与补偿项设计}
\justifying
\begin{itemize}
    \item 曲率平滑项:$J_{\text{curvature}} = \sum_k (\theta_{k+1} - \theta_k)^2$
    \item 水流补偿项:
    \[
    J_{\text{current}} = \sum_k (V_{\text{current}} \cos \psi \cdot e_x(k) + V_{\text{current}} \sin \psi \cdot e_y(k))^2
    \]
    \item 禁区惩罚项:$J_{\text{restricted}} = \lambda_5$,若路径进入禁区,否则为 0
\end{itemize}
\end{frame}

% === 第8页:控制输入约束条件 ===
\begin{frame}{控制输入约束条件}
\justifying
控制输入施加如下约束,确保系统稳定与安全性:
\[
\begin{aligned}
T_{\text{thrust,min}} \leq T_{\text{thrust}}(k) &\leq T_{\text{thrust,max}} \\
T_{\text{lateral,min}} \leq T_{\text{lateral}}(k) &\leq T_{\text{lateral,max}} \\
\tau_{\text{yaw,min}} \leq \tau_{\text{yaw}}(k) &\leq \tau_{\text{yaw,max}}
\end{aligned}
\]
\end{frame}

% === 第9页:状态误差权重矩阵 Q(k) 的动态调整机制 ===
\begin{frame}{状态误差权重矩阵 Q(k) 的动态调整机制}
\justifying
每周期根据当前速度状态动态调整误差矩阵 $Q(k)$:
\[
Q(k) = \text{diag} \left(1 + \alpha(u^2 + v^2), 1 + \beta(u^2 + v^2), \frac{1}{1 + \gamma(u^2 + v^2)}, 1, 1, 1 \right)
\]
调整逻辑:
\begin{itemize}
    \item $X,Y$ 方向误差惩罚随速度增大加强;
    \item 航向角误差惩罚随速度增大减弱;
    \item $u,v,\omega$ 权重固定,用于保障稳定性。
\end{itemize}
该机制提升了高速状态下的位置控制精度与低速下的航向灵活性。
\end{frame}
