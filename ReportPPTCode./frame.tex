

\begin{frame}{三层次路径规划与控制系统架构}
    \justifying
    本研究围绕 USV 在复杂水域中实现自主、高效、鲁棒路径航行的目标,设计了一套“\textbf{全局路径规划 — 局部轨迹优化 — 动态障碍避障}”三层次路径规划与控制系统。各层功能如下:
    
    \vspace{0.2em}
    \begin{itemize}
      \item \textbf{全局路径规划层(MD-A*)}:基于预设地图信息,构建障碍图模型,引入贴边性启发式与绕行惩罚机制,实现全局高效路径生成。
      
      \item \textbf{局部轨迹优化层(AN-MPC)}:结合 Fossen 简化模型与状态反馈,设计自适应加权 MPC 控制器,实现路径跟踪、能耗与速度平稳性的动态权衡。
      
      \item \textbf{动态障碍避障层(展望)}:引入行为识别与策略切换机制,集成改进型多智能体强化学习方法,提升对动态目标的响应能力与智能调整效果。
    \end{itemize}
    
    \vspace{0.2em}
    系统采用感知-规划-控制闭环架构,支持模块间协同运行与实时动态调度,具备良好的工程扩展性。
    \end{frame}
    