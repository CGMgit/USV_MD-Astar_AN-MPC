\section{MD-A* 路径规划方法}

% 第一页:方法概述
\begin{frame}{方法概述:改进型 A* 算法(MD-A*)}
    \justifying
    针对传统 A* 算法在水域路径规划中存在路径保守、贴边性差的问题,本文提出融合几何启发式与边界偏好估计的改进方法。

    \vspace{0.8em}
    本研究构建的 \textbf{MD-A*(Modified Distance A*)} 算法在连边建模中引入“绕行代价”和“穿越长度补偿”两类路径代价因素,同时在启发式函数中融入边界结构感知机制,在保持启发式搜索最优性的前提下,引导路径搜索更贴近可行区域边界。

    \vspace{0.5em}
    方法适用于障碍密集、水道不规则等海港场景,具备良好的工程解释性与可行性。
\end{frame}

% 第二页:直观思想演示图
\begin{frame}[plain]{直观思想演示图1}
  \begin{center}
    \includegraphics[width=0.7\textwidth]{Image/MD-Astar思想演示图.png}
    \captionof{figure}{1 “类贴边性”直观思想对比示意图}
  \end{center}
\end{frame}


% 第三页:权重建模
\begin{frame}{权重建模:融合绕行代价与穿越补偿机制}
    \justifying
    综合考虑欧几里得距离、绕行代价及穿越补偿,定义连边 $(P_i, P_j)$ 的权重为:
    \[
    \text{weight}(P_i, P_j) = d_{\text{euclid}}(P_i, P_j) + \lambda d_{\text{绕行}}(P_i, P_j) - \mu d_{\text{穿越}}(P_i, P_j)
    \]
    其中:
    \begin{itemize}
      \item $d_{\text{euclid}}$:欧几里得距离;
      \item $d_{\text{绕行}}$:对绕开障碍最小路径的估算;
      \item $d_{\text{穿越}}$:连线实际穿越障碍物的长度;
      \item $\lambda \geq \mu \geq 1$:分别为绕行代价与穿越补偿的权重。
    \end{itemize}
    特别地,当边未穿越任何障碍时,$d_{\text{绕行}} = d_{\text{穿越}} = 0$,退化为欧几里得权重。
\end{frame}

% 第四页:启发式函数设计与贴边性引导
\begin{frame}{启发式函数设计与贴边性引导}
    \justifying
    为在路径搜索中兼顾启发效率与路径合理性,本文对传统 A* 启发函数进行设计改进:
    \[
    h(n) = d_{\text{euclid}}(n, t) + \min_{(n,m)\in E} \left( d_{\text{绕行}}(n,m) - d_{\text{穿越}}(n,m) \right)
    \]
    其中,第一项衡量当前节点与目标点的基础距离,第二项表达与边界之间的贴边偏好估计。

    \vspace{0.5em}
    \textbf{可接受性}:$h(n)$ 不高估真实代价;
    \textbf{一致性}:满足 $h(n) \leq c(n,m) + h(m)$,确保 A* 的收敛性与最优性。
\end{frame}


