\section{展望:策略切换机制}

% 第一页:动态避障机制设计概述
\begin{frame}{动态障碍物避障机制设计概述}
\justifying
在复杂动态海洋环境中,水面无人艇(USV)经常面临其他船只、漂浮物等动态障碍物干扰。为提升系统在动态环境下的航行安全性与适应能力,本文提出一种结合\textbf{行为识别}与\textbf{策略切换}机制的动态避障方法,并嵌入于整体三层控制架构中。

\vspace{0.5em}
该方法支持智能体根据障碍物运动特性自动选择避障策略:\textbf{可预测运动}使用基于物理建模的轨迹预测与规避方法;\textbf{不可预测运动}则启用深度强化学习(DRL)模块,实现自适应避障与路径优化。
\end{frame}

% 第二页:直观思想演示图
\begin{frame}[plain]{直观思想演示图3}
\begin{center}
    \includegraphics[height=0.7\textwidth]{Image/策略切换演示图.png}
    \captionof{figure}{3 动态避障策略切换思想演示图}
\end{center}
\end{frame}

% 第三页:行为识别与策略切换机制
\begin{frame}{行为识别与策略切换机制}
\justifying
为判断障碍物是否符合固定方向均匀加速模型,系统使用如下二阶差分指标:
\[
\Delta V = \frac{V_{t+1} - V_t}{\Delta t}, \quad \Delta\theta = \frac{\theta_{t+1} - \theta_t}{\Delta t}
\]
当 $|\Delta V|$ 与 $|\Delta\theta|$ 同时低于阈值,认为该障碍物运动可预测,选择物理预测策略;否则启用深度强化学习模块。
\end{frame}

% 第四页:可预测运动下的物理建模策略
\begin{frame}{可预测运动下的物理建模策略}
\justifying
对于判定为可预测的障碍物,其未来轨迹可通过牛顿运动模型近似计算。障碍物在 USV 船体坐标系下的速度与加速度变换为:
\[
\mathbf{V}_b = R(\theta)^T \cdot \mathbf{V}_g, \quad \mathbf{a}_b = R(\theta)^T \cdot \mathbf{a}_g
\]
其中 $R(\theta)$ 为从全局坐标系到船体坐标系的旋转矩阵。

未来位置预测公式:
\[
p_{\text{obs}}(t + \Delta t) = p_{\text{obs}}(t) + \mathbf{V}_b \cdot \Delta t + \frac{1}{2} \mathbf{a}_b (\Delta t)^2
\]
该方法计算负担轻,适用于短时域规避。
\end{frame}

% 第无页:不可预测运动下的强化学习策略
\begin{frame}{不可预测运动下的强化学习策略}
\justifying
若障碍物运动不符合物理模型,USV 启用基于深度强化学习(DRL)的智能避障模块。通过与环境交互,智能体学习最优行为策略以应对高动态性。

奖励函数设计考虑以下目标:
\begin{itemize}
    \item 成功避障率(避免碰撞);
    \item 控制能耗(惩罚推力波动);
    \item 航迹平滑性(抑制角速度抖动);
    \item 任务完成时间(提高效率)。
\end{itemize}
强化学习模型使用深度神经网络,以状态为输入,输出动作策略。
\end{frame}

% 第六页:避障策略分层结构
\begin{frame}{避障策略的分层控制结构}
\justifying
整个动态避障系统采用三层结构:
\begin{itemize}
    \item \textbf{顶层(行为识别模块)}:实时判断障碍物运动模式;
    \item \textbf{中层(策略切换器)}:根据判断结果选择“物理建模”或“强化学习”路径;
    \item \textbf{底层(控制执行器)}:生成控制指令,调整航向与速度。
\end{itemize}
该架构在兼顾低计算开销与高复杂性适应能力之间实现平衡。
\end{frame}


