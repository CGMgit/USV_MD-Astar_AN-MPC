\section{研究背景与问题定义}

\begin{frame}{研究背景(一)技术发展与政策支持}
    \justifying
    随着智能感知、人工智能与控制理论的不断融合,水面无人艇(Unmanned Surface Vehicle, USV)作为具备自主导航、环境感知与远程通信能力的新型平台,正被广泛应用于海洋测绘、环境监测、灾害预警与水面巡逻等任务场景。
    
    \vspace{0.8em}
    USV 拥有低成本、高效率、危险替代性强等突出优势,成为“智能海洋”建设中的关键组成部分。政策层面,《产业结构调整指导目录(2019年)》明确提出支持智能船舶与无人系统的发展,为相关研究提供了良好的政策导向和产业支撑。
    \end{frame}
    
\begin{frame}{研究背景(二)面对挑战与创新动机}
\justifying
尽管 USV 技术发展迅速,但其在实际水域中的路径规划仍面临典型挑战。尤其在存在大量静态非凸障碍物的环境下,传统 A* 算法在建图与启发式设计中未显式考虑“贴边性”信息,其路径往往趋于保守,远离障碍物边缘,导致航行距离延长、效率下降。

\vspace{0.8em}
此外,路径规划与控制器之间缺乏一致性联动,路径在实际跟踪过程中常出现偏差,影响任务可靠性。为此,本研究在 A* 框架中引入“障碍物边界距离感知”机制,通过改进代价函数与启发策略,使路径具备对边界区域的偏好选择特性。同时结合 AN-MPC 控制器进行局部优化,构建感知、规划与控制一体化的多层路径优化体系。
\end{frame}
